% Currently this document is written in English
% !TeX encoding = UTF-8
% !TeX spellcheck = en_US

%Ensure that all odl school LaTeX habits are remarked
\RequirePackage[l2tabu, orthodox]{nag}

%German: remove "english"
\documentclass[english,utf8,biblatex,noeditorial]{lni-tex/lni}
% Nice tables using \toprule, \midrule, \bottomrule
\usepackage{booktabs}
% subfigures and subcaptions
\usepackage{subcaption}

%% Begin: Drawings
% use standalone and tikz for high-fid. drawings

% standalone package and config
\usepackage{standalone} % For pre-compiled pictures
\standaloneconfig{mode=buildnew} % only build image if source file is newer

% tikz package and config
\usepackage{tikz} % For Tikz pictures
\usetikzlibrary{
  positioning,
  fit,
  arrows,
  calc,
  backgrounds
}
% pgfplots
\usepackage{pgfplots}
\pgfplotsset{compat=1.15} % set pgfplots compatibility to version 1.16
%% End: Drawings

%% Begin: Biblatex

%for easy quotations: \enquote{text}, also required by biblatex
\usepackage{csquotes}
% biblatex is included with LNI-class option: `biblatex`, only set bibliography-file:
\bibliography{paper}

% Clear fields we do not need
\iffalse
\AtEveryBibitem{%
  \ifentrytype{article}{%
  }{%
    \clearfield{doi}%
    \clearfield{issn}%
    \clearfield{url}%
    \clearfield{urldate}%
  }%
  \ifentrytype{inproceedings}{%
  }{%
    \clearfield{doi}%
    \clearfield{issn}%
    \clearfield{url}%
    \clearfield{urldate}%
  }%
}
\fi
%% End: Biblatex

%% Begin: lstlistings

% Scala highlighting
\lstdefinelanguage{scala}{
  morekeywords={%
    abstract,case,catch,class,def,do,else,extends,
    false,final,finally,for,forSome,if,implicit,import,lazy,
    match,new,null,object,override,package,private,protected,
    return,sealed,super,this,throw,trait,true,try,type,
    val,var,while,with,yield},
  otherkeywords={=>,<-,<\%,<:,>:,\#,@},
  sensitive=true,
  morecomment=[l]{//},
  morecomment=[n]{/*}{*/},
  morestring=[b]",
  morestring=[b]',
  morestring=[b]"""
}[keywords,comments,strings]

% configuration of lstlisting
\lstset{%
	xleftmargin=0.5cm, % expected by LNI
    captionpos=b,      % expected by LNI
    fontadjust=true,
    columns=[c]fixed,
    keepspaces=true,
    tabsize=2,
    basicstyle=\renewcommand{\baselinestretch}{0.95}\ttfamily,
    commentstyle=\itshape,
    keywordstyle=\bfseries,
    mathescape=true,
    escapechar=§,
}

% macro for inline code
\newcommand{\code}[1]{\lstinline[flexiblecolumns=true,basicstyle=\renewcommand{\baselinestretch}{0.95}\ttfamily]{#1}}

%% End: lstlistings

%% Begin: Acronyms
\usepackage[acronym]{glossaries}
\glsdisablehyper

% define acronyms here:
\newacronym[longplural={order dependencies}]{od}{OD}{order dependency}
\newacronym[longplural={order compatibility dependencies}]{ocd}{OCD}{order compatibility dependency}

% define special names here (we do not create a glossary, so no descriptions are required)
\newglossaryentry{dactor}{name={Dactor},plural={Dactors},description={}}
\newglossaryentry{functor}{name={Functor},plural={Functors},description={}}
\newglossaryentry{relation}{name={relation},plural={relations},description={}}
%% End: Acronyms


%% Begin: Macros
\newcommand{\order}{\textsc{order}}
\newcommand{\fastod}{\textsc{fastod}}
\newcommand{\ocddiscover}{\textsc{OCDDiscover}}
\newcommand{\dodo}{\textsc{DODO}} % Distributed Order dependency Discovery Optimization
%% End: Macros


%% correct bad hyphenation here
\hyphenation{net-works semi-conduc-tor}


% Start of page count 
% ----------------------- filled out by publisher/editor
\startpage{1}
\editor{Vorname Nachname et al.}
\booktitle{Konferenztitel}
% -----------------------

\author[Sebastian Schmidl \and Juliane Waack]{%
Sebastian Schmidl, Juliane Waack\footnote{Hasso-Plattner-Institut, University of Potsdam, Prof.-Dr.-Helmert-Str. 2-3, 14482 Potsdam, \email{{sebastian.schmidl,juliane.waack}@student.hpi.de}}
}
\title[Distributed Order Dependency]{Distributed Order Dependency}

\begin{document}
\maketitle

% Set to number of authors!
% Authors use each one footnote counter, set this to align the remaining ones.
% E.g. 2 authors --> set to 2, next footnote will be 3
\setcounter{footnote}{2}

\begin{abstract}
  Abstract goes here.
\end{abstract}

\begin{keywords}
Actor Model \and Akka \and Distributed Computing \and Parallelization
\end{keywords}

% Contents
% --------
% !TeX root = paper.tex
% !TeX encoding = UTF-8
% !TeX spellcheck = en_US

% Outline of this paper
        
% !TeX root = ../paper.tex
% !TeX encoding = UTF-8
% !TeX spellcheck = en_US

\section{Introduction}\label{sec:intro}

   \subsection{Motivation}	
	%TODO What are Order Dependencies? Give full definition?
	%	$p_X \succcurlyeq p_Ys$	
	
	Uses for ODs:
   \begin{itemize}
		\item Query Optimization
		\item Data quality (integrity constraints)
		\item Index selection
	\end{itemize}
	
	Challenges:
	\begin{itemize}
		\item Inference co-NP-complete \citep{ginsburg}
		\item best known algorithms between $O(n!)$ \citep{consonni} and $O(2^n)$ \citep{szlichta:discovery} with n being the number of attributes 
		\item use distribution to improve runtime 
	\end{itemize}		  
  

%% !TeX root = ../paper.tex
% !TeX encoding = UTF-8
% !TeX spellcheck = en_US

\section{Related Work}\label{sec:related_work}
  The concept of order dependencies was first introduced in the context of database systems by \citeauthor{ginsburg}~\cite{ginsburg} as \textit{point-wise ordering}.
  \citeauthor{ginsburg}'s definition specified that a set of columns orders another set of columns.

  \citeauthor{szlichta:fundamentals}~\cite{szlichta:fundamentals} later introduced another definition for order dependencies, which was used by the following research in this area \cite{consonni, langer, szlichta:discovery} and is also the basis of this work.
  It differentiates from the point-wise ordering by considering list of columns instead of sets.
  This leads to a lexicographical ordering of tuples, as by the \texttt{order by} operator in SQL.

%TODO go into more detail about how the algorithms work?
\citeauthor{langer} published the first algorithm to find all minimal order dependencies based on \citeauthor{szlichta:fundamentals}'s definition. 
It is called \texttt{order} and has a factorial runtime.\\
When publishing their own algorithm \texttt{fastod}, \citeauthor{szlichta:discovery} showed, that \citeauthor{langer}'s definition of minimality was incomplete. 
\texttt{fastod} has an exponential worst time complexity \citep{szlichta:discovery}.\\
When introducing the algorithm \texttt{ocddiscover}, \citeauthor{consonni} claimed, that a provided implementation of \texttt{fastod} did not find all order dependencies. 
\texttt{ocddiscover} has a factorial runtime but was shown to outperform \texttt{fastod} on some datasets. 
It is also capable of multithreating, which makes it easy to adapt to a distributed system.
This was the state of published research when we started on this project, which is why we based our own algorithm on \texttt{ocddiscover}. \\
Since then, Szlichta et al have published an Errata Note demonstrating an error in \citeauthor{consonni}'s proof of minimality and explaining, how \citeauthor{consonni} misinterpreted \texttt{fastod}'s results.
Since the goal of our seminar was the distribution of an already existing algorithm with a focus on reliability and scalability, however, we decided to continue working on \texttt{ocddiscover}.


%\input{sections/concept}

%\input{sections/framework}

%\input{sections/experiments}

%% !TeX root = ../paper.tex
% !TeX encoding = UTF-8
% !TeX spellcheck = en_US

\section{Conclusion}\label{sec:conclusion}

  In this work, we presented \dodo{}, a scalable, fault-tolerant, distributed \gls{od} discovery algorithm implemented on top of the Akka toolkit.
  \dodo{} can be deployed on a single node or in a cluster.
  It makes use of work stealing to distribute load and to deal with dynamic cluster sizes.
  A state replication protocol ensures fault-tolerance in the case of message loss and node failures.
  We based \dodo{}'s \gls{od} discovery approach on \ocddiscover{} introduced by \citeauthor{consonni}~\cite{consonni}.
  Our approach outperforms \ocddiscover{} by about a factor of two on a single node setup and we can even scale out across several nodes.
  Our experiments show that distributing an \gls{od} discovery algorithm across nodes greatly reduces computation time of the algorithm.
  \dodo{} in cluster mode with eight nodes achieves a four times speedup compared to a single node \dodo{} setup.
  This proves that employing distributed algorithms to aid \gls{od} discovery reduces computation times and increases their robustness.
  This opens the way for using \gls{od} discovery to find hidden dependencies in structured big data.

  \dodo{} currently uses \citeauthor{consonni}'s definition of minimality.
  As \citeauthor{szlichta:errata}~\cite{szlichta:errata} have pointed out, this does not find all minimal \glspl{ocd} in the dataset.
  A valuable next step would therefore be the change of the discovery algorithm to correctly prune the search space according to \cite{szlichta:errata}.
  Additionally, we currently do not remove duplicates from the results and we do not automatically merge the result sets from different nodes.
  Duplicates are produced when \dodo{} recovers from node failures and some \gls{ocd} candidates must be re-checked on another node to prevent data loss.

  \dodo{} is implemented using the actor programming model.
  It provides a dynamic computation aspect.
  This can be used to run different \gls{od} discovery approaches in a single system at the same time and to exchange information between them.
  This could decrease the computation time further, because the exchanged information can be used to prune the search spaces for the different approaches dynamically.
  Future work has to evaluate if the benefits of this approach outweigh their disadvantages, such as increased communication overhead and running two approaches simultaneously.
  Our approach does not make use of this aspect of the actor model so far.

  Finally, we did not yet perform micro-benchmarks with \dodo{}.
  We still want to
  (i) test the memory boundaries of \dodo{},
  (ii) evaluate different settings for batch-sizes and timeout durations, and
  (iii) we want to have a look at the impact of the work stealing and state replication protocols on the performance of our algorithm.

% did not fit in:
%\begin{itemize}
%  \item interesting questions: how to proove we have founbd all edgecases / formalize protocols
%  \item Find and fix work stealing edge cases
%\end{itemize}
% --------

% bibliography
\printbibliography

\end{document}
