% !TeX root = ../paper.tex
% !TeX encoding = UTF-8
% !TeX spellcheck = en_US

\section{Conclusion}\label{sec:conclusion}

  In this work, we presented \dodo{}, a scalable, fault-tolerant, distributed \gls{od} discovery algorithm implemented on top of the Akka toolkit.
  \dodo{} can be deployed on a single node or in a cluster.
  It makes use of work stealing to distribute load and to deal with dynamic cluster sizes.
  A state replication protocol ensures fault-tolerance in the case of message loss and node failures.
  We based \dodo{}'s \gls{od} discovery approach on \ocddiscover{} introduced by \citeauthor{consonni}~\cite{consonni}.
  Our approach outperforms \ocddiscover{} by about a factor of two on a single node setup and we can even scale out across several nodes.
  Our experiments show that distributing an \gls{od} discovery algorithm across nodes greatly reduces computation time of the algorithm.
  \dodo{} in cluster mode with eight nodes achieves a four times speedup compared to a single node \dodo{} setup.
  This proves that employing distributed algorithms to aid \gls{od} discovery reduces computation times and increases their robustness.
  This opens the way for using \gls{od} discovery to find hidden dependencies in structured big data.

  \dodo{} currently uses \citeauthor{consonni}'s definition of minimality.
  As \citeauthor{szlichta:errata}~\cite{szlichta:errata} have pointed out, this does not find all minimal \glspl{ocd} in the dataset.
  A valuable next step would therefore be the change of the discovery algorithm to correctly prune the search space according to \cite{szlichta:errata}.
  Additionally, we currently do not remove duplicates from the results and we do not automatically merge the result sets from different nodes.
  Duplicates are produced when \dodo{} recovers from node failures and some \gls{ocd} candidates must be re-checked on another node to prevent data loss.

  \dodo{} is implemented using the actor programming model.
  It provides a dynamic computation aspect.
  This can be used to run different \gls{od} discovery approaches in a single system at the same time and to exchange information between them.
  This could decrease the computation time further, because the exchanged information can be used to prune the search spaces for the different approaches dynamically.
  Future work has to evaluate if the benefits of this approach outweigh their disadvantages, such as increased communication overhead and running two approaches simultaneously.
  Our approach does not make use of this aspect of the actor model so far.

  Finally, we did not yet perform micro-benchmarks with \dodo{}.
  It would be interesting to further evaluate
  (i) the memory boundaries of \dodo{},
  (ii) the different settings for batch-sizes and timeout durations, and
  (iii) the impact of the work stealing and state replication protocols on the performance of our algorithm.

% did not fit in:
%\begin{itemize}
%  \item interesting questions: how to proove we have founbd all edgecases / formalize protocols
%  \item Find and fix work stealing edge cases
%\end{itemize}