
\section{Example}
\label{sec:lniconformance}
This example is from the original LNI documentation and therefore in German:

Referenzen mit dem richtigen Namen (\textit{Table}, \textit{Figure}, ...):
\Cref{sec:intro} zeigt Demonstrationen der Verbesserung von GitHub-LNI gegenüber der originalen Vorlage.
\Cref{sec:lniconformance} zeigt die Einhaltung der Richtlinien durch einfachen Text.

Referenzen sollten nicht direkt als Subjekt eingebunden werden, sondern immer nur durch Authorenanganben:
Beispiel: \Citet{AB00} geben ein Beispiel, aber auch \citet{Az09}.
Hinweis: Großes C bei \texttt{Citet}, wenn es am Satzanfang steht. Analog zu \texttt{Cref}.

\Cref{fig:demo} zeigt eine Abbildung.

\begin{figure}
  \centering
  Hier sollte die Graphik mittels \texttt{includegraphics} eingebunden werden.

  %\includegraphics[width=.8\textwidth]{filename}
  \caption{Demographik}
  \label{fig:demo}
\end{figure}

\Cref{tab:demo} zeigt eine Tabelle.

\begin{table}
  \centering
  \begin{tabular}{lll}
    \toprule
    Überschriftsebenen & Beispiel & Schriftgröße und -art \\
    \midrule
    Titel (linksbündig) & Der Titel \ldots & 14 pt, Fett\\
    Überschrift 1 & 1 Einleitung & 12 pt, Fett\\
    Überschrift 2 & 2.1 Titel & 10 pt, Fett\\
    \bottomrule
  \end{tabular}
  \caption{Die Überschriftsarten}
  \label{tab:demo}
\end{table}
               
Die LNI-Formatvorlage verlangt die Einrückung von Listings vom linken Rand.
\Cref{L1} zeigt uns ein Beispiel, das mit Hilfe der \texttt{lstlisting}-Umgebung realisiert ist.
Referenz auf \code{print("Hello World!")} in \cref{lst:l2}.

% activate the language and add caption/label
% use `§` to add latex stuff, such as labels
% math works inside of listings like in latex with `$`
\begin{lstlisting}[caption=Beschreibung, label=L1, language=Scala]
/**
  * Hello World application!
  */
object Hello { 
  def main(args: Seq[String]): Unit = {
    println("Hello World!") §\label{lst:l2}§
  }
  def math: Unit = $\frac{\sqrt{1 - x^2}}{x_i + \lambda} - \infty$
}
\end{lstlisting}

Die korrekte Einrückung und Nummerierung für Formeln ist bei den Umgebungen \texttt{equation} und \texttt{align} gewährleistet.

\begin{equation}
  1=4-3 + x
\end{equation}
und
\begin{align}
  2=7-5\\
  3=2-1
\end{align}
