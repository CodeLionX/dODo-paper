% !TeX root = ../paper.tex
% !TeX encoding = UTF-8
% !TeX spellcheck = en_US

\section{Introduction}\label{sec:intro}
%TODO Some sort of introductory sentence about big data 
\glspl{od} describe relationships between lists of columns in datasets and are closely related to functional dependencies. 
An \gls{od} exists when a table being ordered by one list of attributes causes the table to also be ordered by another list of attributes.
Knowledge of \glspl{od} can help in understanding the semantics behind the data which can be used to ensure and/or improve data quality e.g. by uncovering and applying constraints.
They can also be used for query optimization, e.g. to eliminate unneccessary sorting steps, and aid in index selection.

Some of the challenges in discovering \glspl{od} are the large searchspace, because every combination of attributes could order any other combination of attributes. 
In the process of reducing this search space several defenitions for minimality of \glspl{od} have been proposed \citep{langer}\citep{szlichta:discovery}.
% More specifically, Langer et al disregard ODs with repeated attributes on both sides while Szlichta et al do not.
Nevertheless all currently published algorithms for discovering all minimal \glspl{od} on a dataset have an exponential or factorial runtime. 
Some speed-up can be achieved using parallelization. 
On one machine, this is limited by the number of threads, however.
For better scalability we propose distributing an \gls{od} discovery algorithm over several machines.
To this end we adapted the algorithm \ocddiscover{} using the actor-based Akka framework.

