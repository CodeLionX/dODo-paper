% !TeX root = ../paper.tex
% !TeX encoding = UTF-8
% !TeX spellcheck = en_US

\section{Conclusion}\label{sec:conclusion}

\begin{itemize}
\item Summary
\item real conclusion
\item future work
\begin{itemize}
\item what's not implemented yet (merging of results, deduplication, fixing algorithm)
\item optimize (sort and compare on multiple columns (consider data types or merge values to one or define ordering over indices?))
\item interesting questions: how to proove we have founbd all edgecases / formalize protocols
\item Test memory boundaries, possibly optmize memory consumption (delta protocol?), look at garbage collector overhead
\item Find and fix work stealing edge cases 
\item evaluate timeouts
\item The actor model provides a dynamic aspect. 
This can be used to run multiple approaches in a single system and to exchange information between them. 
The goal of this would be to benefit from the advantages of the different approaches and at the same time decreasing their disadvantages.
Our approach does not make use of this aspect of the actor system so far.
\end{itemize}
\end{itemize}